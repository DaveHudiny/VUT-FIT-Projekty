\documentclass[11pt, a4paper, twocolumn]{article}
\usepackage[utf8]{inputenc}
\usepackage[IL2]{fontenc}
\usepackage[left=1.5cm,top=2.5cm, text={18cm, 25cm}]{geometry}
\usepackage[czech]{babel}
\title{ITY2}
\author{xhudak03}
\date{March 2021}
\usepackage{amsmath, amsthm, amssymb}
\usepackage{times}

\theoremstyle{definition}
\newtheorem{d1}{Definice}
\newtheorem{v1}{Věta}


\begin{document}
\begin{titlepage}
\begin{center}
{\Huge
\textsc{Fakulta informačních technologií\\[0.4em]Vysoké učení technické v~Brně}}\\
\vspace{\stretch{0.382}}
\LARGE
Typografie a publikování -- 2. projekt\\[0.3em]Sazba dokumentů a matematických výrazů
\vspace{\stretch{0.618}}
\end{center}
{\LARGE 2021 \hfill
David Hudák (xhudak03)}
\thispagestyle{empty}
\end{titlepage}

\clearpage
\pagenumbering{arabic} 

\section*{Úvod}

V~této úloze si vyzkoušíme sazbu titulní strany, matematických vzorců, prostředí a dalších textových struktur obvyklých pro technicky zaměřené texty (například rovnice~\eqref{rovnice1}
nebo Definice \ref{def1} na straně \pageref{rovnice}). Rovněž si vyzkoušíme používání odkazů {\fontfamily{pcr}\selectfont \symbol{92}ref} a
{\fontfamily{pcr}\selectfont \symbol{92}pageref}.

Na titulní straně je využito sázení nadpisu podle optického středu s~využitím zlatého řezu. Tento postup byl
probírán na přednášce. Dále je použito odřádkování se
zadanou relativní velikostí 0.4 em a 0.3 em.

V~případě, že budete potřebovat vyjádřit matematickou
konstrukci nebo symbol a nebude se Vám dařit jej nalézt
v~samotném \LaTeX u, doporučuji prostudovat možnosti balíku maker \AmS -\LaTeX.

\section{Matematický text}

Nejprve se podíváme na sázení matematických symbolů
a~výrazů v~plynulém textu včetně sazby definic a vět s~využitím 
balíku {\fontfamily{pcr}\selectfont amsthm}. Rovněž použijeme poznámku pod
čarou s~použitím příkazu {\fontfamily{pcr}\selectfont
\symbol{92}footnote}. Někdy je vhodné
použít konstrukci {\fontfamily{pcr}\selectfont
\symbol{92}mbox\symbol{123}\symbol{125}}, která říká, že text nemá být
zalomen.
\begin{d1}
\label{def1}
Rozšířený zásobníkový automat \textit{(RZA) je de\-finován jako sedmice tvaru $A = (Q, \Sigma, \Gamma, \delta, q_0, Z_0, F)$,
kde:}
\begin{itemize}
    \item $Q$\textit{ je konečná množina} vnitřních (řídicích) stavů,
    \item $\Sigma$ \textit{je konečná} vstupní abeceda,
    \item $\Gamma$ \textit{je konečná} zásobníková abeceda,
    \item $\delta$ \textit{je} přechodová funkce $Q \times (\Sigma \cup \{\epsilon\})\times\Gamma^\ast\rightarrow 2^{Q\times\Gamma^*}$,
    \item $q_0 \in Q$ \textit{je} počáteční stav, $Z_0 \in \Gamma$ \textit{je} startovací symbol zásobníku \emph{a} $F \subseteq Q$ \textit{je množina} koncových stavů.
\end{itemize}

Nechť $P=(Q, \Sigma, \Gamma, \delta, q_0, Z_0, F)$ je rozšířený zásobníkový automat. \textit{Konfigurací} nazveme trojici $(q, w, \alpha) \in Q \times \Sigma^* \times \Gamma^*$, kde $q$ je aktuální stav vnitřního řízení, $w$ je dosud nezpracovaná část vstupního řetězce a $\alpha =
Z_{i_1}Z_{i_2}
\dots Z_{i_k}$
je obsah zásobníku\footnote{$Z_{i1}$ je vrchol zásobníku}.
\subsection{Podsekce obsahující větu a odkaz}
\end{d1}

\begin{d1}
\label{def2}
Řetězec $w$ nad abecedou $\Sigma$ je přijat RZA
$A$ \textit{jestliže $(q_0, w, Z_0)\overset{*}{\underset{A}{\vdash}}(q_F, \epsilon, \gamma)$ pro nějaké $\gamma\in \Gamma^*$ a $q_F \in F$. Množinu $L(A) = \{w$ $|$ $w$ je přijat RZA $A\} \subseteq \Sigma^*$~nazýváme} jazyk přijímaný RZA $A$.
\end{d1}

Nyní si vyzkoušíme sazbu vět a důkazů opět s~použitím
balíku {\fontfamily{pcr}\selectfont amsthm}.
\begin{v1}
\textit{Třída jazyků, které jsou přijímány ZA, odpovídá}
bezkontextovým jazykům.
\end{v1}
\begin{proof}
V~důkaze vyjdeme z~Definice \ref{def1} a \ref{def2}.
\end{proof}

\section{Rovnice a odkazy}
\label{rovnice}

Složitější matematické formulace sázíme mimo plynulý
text. Lze umístit několik výrazů na jeden řádek, ale pak je
třeba tyto vhodně oddělit, například příkazem {\fontfamily{pcr}\selectfont \symbol{92}quad}.
\\
$$\sqrt[i]{x_i^3}\quad \text{kde } x_i \text{ je i-té sudé číslo splňující}\quad x_i^{x_i^{i^2}+2} \leq y_i^{x_i^4}$$

V~rovnici \eqref{rovnice1} jsou využity tři typy závorek s~různou
explicitně definovanou velikostí.

\begin{eqnarray}
\label{rovnice1}
    x &=& \bigg[\Big\{\big[a+b\big]*c\Big\}^d\oplus 2\bigg]^{3/2}\\
    y&=&\lim_{x \rightarrow \infty}\frac{\frac{1}{\log_{10} x}}{\sin^2 x + \cos^2 x} \nonumber
\end{eqnarray}

V~této větě vidíme, jak vypadá implicitní vysázení limity $\lim_{n\rightarrow \infty}f(n)$ v~normálním odstavci textu. 
Podobně je to i s~dalšími symboly jako $\prod^n_{i=1}2^i$ či $\bigcap_{A\in \mathcal{B}}A$.
V~případě vzorců $\lim\limits_{n\rightarrow \infty}f(n)$ a
$\prod\limits^n_{i=1}2^i$ jsme si vynutili méně
úspornou sazbu příkazem {\fontfamily{pcr}\selectfont \symbol{92}limits}.


\begin{equation}
\label{rovnice2}
    \int_b^a g(x)dx\quad =\quad -\int\limits_{a}\limits^{b} f(x)dx
\end{equation}


\section{Matice}
Pro sázení matic se velmi často používá prostředí {\fontfamily{pcr}\selectfont array}
a závorky ({\fontfamily{pcr}\selectfont \symbol{92}left}, {\fontfamily{pcr}\selectfont \symbol{92}right}).

$$
\left(\begin{array}{ccc}
a-b & \widehat{\xi+\omega} & \pi \\
\vec{\mathbf{a}} & \overleftrightarrow{A C} & \hat{\beta}
\end{array}
\right)
= 1 \Longleftrightarrow \mathcal{Q}=\mathbb{R}
$$

$$
A =
\left\|
\begin{array}{cccc}
a_{11} & a_{12} & \cdots & a_{1n} \\
a_{21} & a_{22} & \ldots & a_{2n} \\
\vdots & \vdots & \ddots & \vdots \\
a_{m1} & a_{m2} & \cdots & a_{mn}
\end{array}
\right\|
=
\left| 
\begin{array}{cc}
t & u \\
v & w
\end{array} \right|
= tw - uw
$$

Prostředí {\fontfamily{pcr}\selectfont array} lze úspěšně využít i jinde.

$$ \binom{n}{k} = \left\{
\begin{array}{c l}
0 & \text{pro } k < 0 \text{ nebo } k > n \\
\frac{n!}{k!(n-k)!} & \text{pro } 0 \leq k \leq n
\end{array} \right. $$







\end{document}
