\documentclass[14pt]{beamer}
\usepackage[utf8]{inputenc}
\usepackage[T1]{fontenc}
\usepackage[czech]{babel}
\usetheme{Berlin}
\usecolortheme{dolphin}
\bibliographystyle{czplain}
\renewcommand{\UrlFont}{\ttfamily\small}
\usepackage{url}
\usepackage{calc}
\usepackage{graphicx}
\usepackage{svg}
\usepackage{amsmath}
\usepackage{amssymb}
\usepackage{mathrsfs}
\usepackage{algorithm}
\usepackage{algorithmic}


\title{Hashovací tabulka}
\author{David Hudák}
\date{xhudak03}
\begin{document}

\maketitle
\section{Hashovací tabulky}
\subsection{Motivace}
\begin{frame}{Motivace 1/2 -- úvod}
    \begin{itemize}
        \item U některých problémů:
        \begin{itemize}
            \item Odhadovaná velikost přiměřená
            \item Preferována rychlost
            \item Častý přístup k položkám
        \end{itemize}
        \item Příznivá časová složitost
        \item Dnes běžné implementace ve standardních knihovnách
    \end{itemize}
\end{frame}
\begin{frame}{Motivace 2/2 -- reálné využití}
    \begin{itemize}
        \item Slovníky\cite{stack}
        \begin{itemize}
            \item Ukládání proměnných v překladači
            \item Klasické využití slovníku
        \end{itemize}
        \pause
        \item Databáze
        \begin{itemize}
            \item Rychlé vyhledání informace na základě klíče
        \end{itemize}
        \pause
        \item Přesah do kryptografie
    \end{itemize}
\end{frame}
\subsection{Základní princip}

\begin{frame}{Teoretický základ}
    \begin{itemize}
        \item Hashovací funkce
        \begin{itemize}
            \item Schopna vypočítat index ze vstupního klíče
            \item Ideálně vytvářející minimum kolizí -- problém:
            \begin{itemize}
                \item Pro $n$ prvků pro mapování a $m$ políček v tabulce $m^n$ možných funkcí
            \end{itemize}
        \end{itemize}
        \item Hashovací tabulka -- pole struktur
        \begin{itemize}
            \item Klíč
            \item Obsazenost
            \item Data
            \item Ukazatel na další prvek (nepovinné)
        \end{itemize}
    \end{itemize}
\end{frame}
\section{Kolize}
\subsection{Kolize}
\begin{frame}{Řešení kolizí}
    \begin{itemize}
        \item Kolize -- dva různé prvky se stejným indexem (synonyma)
        \item Implicitní řešení kolizí
        \begin{itemize}
            \item Kolizní prvek získává adresu z opakovaného výpočtu hashovací funkce
        \end{itemize}
        \item Explicitní řešení kolizí
        \begin{itemize}
            \item Každý element tabulky lineárním seznamem
        \end{itemize}
        \item Od zvolené možnosti se odvíjí implementace vyhledávání, vkládání i mazání (u implicitního zřetězení mazání komplikované)
    \end{itemize}
\end{frame}

\subsection{Ukázka}
\begin{frame}{Ukázka 1/3}
\begin{itemize}
    \item Funkce sčítá hodnoty všech znaků (pouze ukázková)
    \begin{equation*}
        index = \sum_{i=0}^{n}key[i] \text{ mod } m
    \end{equation*}
    \begin{itemize}
        \item $key[i]$ -- hodnota řetězce na indexu $i$
        \item $index$ -- index, kam se má přistupovat v~tabulce
        \item $n$ -- počet všech prvků klíče
        \item $m$ -- počet prvků tabulky
    \end{itemize}
\end{itemize}
\end{frame}

\begin{frame}{Ukázka 2/3}
\begin{itemize}
    \item Velikost tabulky 100
    \item Vkládáme slovo blecha
    \begin{itemize}
        \item $index = 7$
    \end{itemize}
    \item Vkládáme slovo chleba
    \begin{itemize}
        \item $index = 7$
        \item \textbf{Kolize!}
    \end{itemize}
\end{itemize}
\end{frame}
\begin{frame}{Ukázka 3/3}
    \begin{figure}
        \centering
        \includesvg[inkscapelatex=false, scale=0.6, keepaspectratio]{ukazka.svg}
        \label{fig:my_label}
    \end{figure}
\end{frame}
\section{Pseudokód}
\subsection{Vyhledávání v tabulce}
\begin{frame}{Vyhledávání při explicitním zřetězení}
    \begin{algorithm}[H]
        \begin{algorithmic}[1]
            \STATE $index = hash(key)$
            \IF {$key == table[index].key$} 
                \RETURN $True$
            \ELSE
                \RETURN $inList(table[index].list, key)$
            \ENDIF
        \end{algorithmic}
        \label{alg:seq}
    \end{algorithm}
\end{frame}
\section{Složitost}
\subsection{Složitost}
\begin{frame}{Složitost}
    \begin{itemize}
        \item Vkládání, vyhledávání, mazání
        \begin{itemize}
            \item $O(1)$ -- nejpříznivější; pouze čas výpočtu hashovací funkce
            \item $O(n)$ -- nejhorší; hashovací tabulka transformována do lineárního seznamu (moc kolizí)
        \end{itemize}
        \item Důležitá kvalita hashovací funkce
        \item Velký vliv velikosti hashovací tabulky
    \end{itemize}
\end{frame}
\section{Reference}
\subsection{Reference}
\begin{frame}{Reference}
\fontsize{9pt}{10}\selectfont
\bibliography{bib}    
\end{frame}


\end{document}
