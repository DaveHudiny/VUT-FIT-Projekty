\documentclass[11pt, a4paper]{article}
\usepackage[utf8]{inputenc}
\usepackage[IL2]{fontenc}
\usepackage[left=2cm,top=3cm, text={17cm, 24cm}]{geometry}
\usepackage[vietnamese=nohyphenation]{hyphsubst}
\usepackage[czech, vietnamese.licr]{babel}
\title{ITY2}
\author{xhudak03}
\date{March 2021}
\usepackage{times}
\usepackage{multirow}
\usepackage[unicode]{hyperref}
\bibliographystyle{czechiso}


\begin{document}
\selectlanguage{czech}
\begin{titlepage}
\begin{center}
{\Huge
\textsc{Vysoké učení technické v~Brně\huge{\\[0.4em]Fakulta informačních technologií}}}\\
\vspace{\stretch{0.382}}
\LARGE
Typografie a publikování -- 4. projekt\\[0.3em]\huge Stručná historie \LaTeX u
\vspace{\stretch{0.618}}
\end{center}
{\LARGE \today \hfill
David Hudák}
\thispagestyle{empty}
\end{titlepage}

\clearpage
\pagenumbering{arabic} 
\newpage


\section{Název}
\begin{sloppypar}
Při prvním setkání s~prostředím \LaTeX~jsem se já a mí gymnaziální spolužáci
pousmáli nad variantami jeho výslovnosti. Někde se můžeme setkat
s~výslovností \verb|[leitek]| (například v~předmětu IEL), 
avšak převažujícím způsobem výslovnosti (včetně předmětu ITY) je
\verb|[latech]|. Můj středoškolský učitel to zdůvodňuje na svých stránkách\cite{martinek1} způsobem, že se jedná o~zkratku \uv{\textbf{La}mportův \textbf{TeX}}, kde Lamport se vyslovuje tak, jak se píše, a~\textbf{X} ve slově \textbf{TeX} je řeckým písmenem chí ($X$).
\end{sloppypar}
\section{Vznik \LaTeX u}
Jak bylo zmíněno v~odstavci výše, název \LaTeX~se skládá i ze jména Lamport.
Jedná se o~původního autora systému, který vytvořil v~roce 1984 \LaTeX~a
prvního manuálu k~němu z~roku 1986\cite{prvnimanual}. Ten následně předal
údržbu i vývoj Frankovi Mittelbachovi, který v~rámci své společnosti
pokračoval a vyvinul \LaTeX~2e (současná verze) a nyní pracuje na \LaTeX  
3\cite{lamport}.

\section{pdfTeX}
\label{pdf}
Důležitou brněnskou událostí pro \LaTeX byla publikace dizertační práce 
na Masarykově univerzitě publikovaná studentem \selectlanguage{vietnamese}~Hàn Thế Thànhem\selectlanguage{czech}\cite{pdflatex} a vedoucím práce byl současný děkan Fakulty informatiky a bývalý rektor Masarykovy univerzity Jiří Zlatuška\cite{zlatuska}. Tato práce výrazně posunula práci s~\LaTeX em, a to tím způsobem, že nyní již není nutné generovat soubor s~příponou \verb|.dvi|, nýbrž přímo univerzální PDF dokument. Není snad nutno zmiňovat, že právě formát PDF je mnohem šířeji přijímaný než ostatní formáty dokumentů a dává tak možnost prosadit se nejen v~technické sféře.

\section{Aktuální možnosti práce s~\LaTeX em}
S~postupem času a přidáním širších funkcionalit (viz třeba \ref{pdf}) se výrazně rozšířily možnosti práce s~\LaTeX em. Například v~roce 2012 vznikla společnost Overleaf, která vytváří online editor pro editaci a kompilaci \LaTeX u v~rychlém a snadno použitelném prostředí\cite{overleaf}. Z~ostatních možností se dá zmínit například MixTeX\cite{miktex} a TeXmaker\cite{texmaker}.

\section{Současný \LaTeX~a typografie}
V~současné době typografie zůstává velkým a důležitým tématem a vzniká na
toto téma mnoho časopisů. Například v~Česku se dá pořídit časopis
Font\cite{casFont}. Co se týká dostupné literatury ke studiu pro práci
s~\LaTeX em, tak nejsou novější možnosti příliš široké. Doporučit se dá přesto
kniha z~roku 2003 od pana Rybičky\cite{rybicka}.

Typografie nemusí být jen součástí knih, ale je i důležitým kritériem pro tvorbu uživatelských rozhraní. Například bakalářská práce Marka Vyroubala\cite{random} má v~klíčových slovech slovo typografie. Při zadání slova typography do systému Google Scholar se dá narazit na jistě velmi zajímavý článek za 25~\texteuro~od jistého Theo van Leeuwena\cite{dutch} nebo tento další článek o~semiotice\cite{semiotic}.
\newpage
\bibliography{bibliografie}
\end{document}

